\documentclass{article}

\usepackage[paper=a4paper, verbose, centering, margin=1.3in]{geometry}
\usepackage[utf8]{inputenc}
\usepackage{amsmath}
\usepackage{amssymb}
\usepackage{amsthm}
\usepackage{centernot}
\usepackage{graphicx}
\usepackage{amsthm}
\usepackage{mathtools}

\newcommand{\Z}{\mathbb{Z}}
\newcommand{\R}{\mathbb{R}}

\renewcommand{\i}{^{-1}}
\newcommand{\set}[1]{\left\{ #1 \right\}}

\newcommand{\Leech}{\Lambda_{24}}

\newcommand{\header}[1]{\noindent\textsc{#1.} }

\begin{document}

\section*{1. The Sphere Packing Problem}

\subsection*{1.4. $n$-Dimensional Packings}

\header{Definition}
The \textbf{fcc} lattice is the face-centered cubic lattice.

\header{Definition}
A lattice $\Lambda$ has a \textbf{dual} lattice $\Lambda^*$ given by
$$ \Lambda^* := \set{ x : \forall u \in \Lambda, x \cdot u \in \Z }. $$
For example, the dual of the fcc lattice is the \textbf{body-centered cubic} lattice (bcc lattice).
If $A$ is a Gram matrix for $\Lambda$, then $\Lambda^*$ has Gram matrix $A\i$.

Why is is finding dense packings in $n$-dimensions interesting?
\begin{enumerate}
  \item
  Interesting problem in pure geometry.
  Hilbert mentioned it in 1900 in his ist of open problems [Hil1], [Mil5].

  \item
  Has (sometimes unexpected) connections to other branches of mathematics.
  For example, the densest lattice packings in up to 8 dimensions belong to the families $A_n, D_n, E_n$ and the corresponding Coxeter-Dynkin diagrams turn in several seemingly unrelated areas.

  \item
  The Leech lattice in 24 dimensions, $\Leech$, has mysterious connections with hyperbolic geometry, Lie algebras, and the Monster simple group.

  \item
  There are direct applications of lattice packings to number theory e.g. solving Diophantic equations and the ``geometry of numbers.'' [Cas2], [Gru1], [Gru1a], [Han3], [Hla1], [Hla3], [Kel1], [Min4], [Min6].
  (See Section 2.3.)

  \item
  There are practical applications of sphere packings to digital communications (see Chapter 3).

  \item
  2- and 3-dimensional spheres have many practical applications in general e.g. positioning optical fibers in the cross-section of a cable [Kin1], chemistry, physics, antenna design, X-ray tomography, and statistical analysis (on spheres).

  \item
  $n$-dimensional packings may be used in the numerical evaluation of integrals, either on the surface of a sphere in $\R^n$ or its interior. (See Section 3.2.)

  \item
  Dual theory and superstring theory in physics have made use of $E_8$ and $\Leech$ lattices and their related Lorentzian lattices in dimensions 10 and 26 discussed in Chapters 26 and 27.

\end{enumerate}

\subsection*{1.5. Summary of Results in Sphere Packing}

TODO: not sure if this is relevant enough to read. I just want the introductory math to write about.

\section*{2. The Kissing Number Problem}

% TODO: hello

\end{document}

\documentclass{article}

\usepackage[paper=a4paper, verbose, centering, margin=1.3in]{geometry}
\usepackage[utf8]{inputenc}
\usepackage{amsmath}
\usepackage{amssymb}
\usepackage{amsthm}
\usepackage{centernot}
\usepackage{graphicx}
\usepackage{amsthm}
\usepackage{mathtools}

\renewcommand{\tt}[1]{\text{ #1 }}

\newcommand{\Z}{\mathbb{Z}}
\newcommand{\R}{\mathbb{R}}
\newcommand{\C}{\mathcal{C}}
\newcommand{\F}{\mathbb{F}}

\renewcommand{\i}{^{-1}}
\newcommand{\sse}{\subseteq}
\renewcommand{\ss}{\subset}
\newcommand{\degrees}{^\circ}
\newcommand{\ass}[1]{\left( #1 \right)}
\newcommand{\set}[1]{\left\{ #1 \right\}}

\newcommand{\Leech}{\Lambda_{24}}
\newcommand{\Golay}{\mathcal{C}_{24}}
\newcommand{\PL}{\textit{PL}}

\newcommand{\header}[1]{\noindent\textsc{#1.} }

\begin{document}

\section*{1. The Golay Code}

\header{Definition}
Let $\F_{23}$ be the finite field of order 23.
Let $\Omega$ be the projective line over $\F_{23}$, $\PL(23)$.
Let $Q$ be the subset of $\Omega$ comprised of the quadratic residues modulo 23
and let $N := \Omega \setminus Q$.
Altogether:
\begin{align*}
  \Omega &:= \PL(23) = \set{ \infty, 0, 1, \dots, 22 } \\
  Q &:= \set{ x^2 : x \in \F_{23} } = \set{ 0, 1, 3, 4, 6, 8, 9, 12, 13, 16, 18 } \\
  N &:= \Omega \setminus Q = \set{ \infty, 5, 7, 10, 11, 13, 15, 17, 19, 20, 21, 22 }
\end{align*}
Define
\begin{align*}
  N_i &:= \begin{cases}
    N - i = \set{ n - i : n \in N } &\tt{if} i \neq \infty \\
    \Omega &\tt{if} i = \infty
  \end{cases}
  \\
  N_S &:= \sum_{i \in S} N_i
\end{align*}
where $i \in \Omega$ and $S \sse \Omega$ and
$N_S$ denotes the symmetric difference of the set $N_i$.

\header{Definition}
The \textbf{binary linear code of length n} is a vector subspace of $\F_2^n$.
The set of elements of the code is called the \textbf{codeword set}.

\header{Definition}
The \textbf{Golay code} is a binary linear code of length 24 with codeword set $\C_{24} \sse \F_2^{24}$ spanned by the 24 vectors $v_i$ with 1s in the places of the elements of $N_i$ and 0s elsewhere.
The codewords are called $\C$-\textbf{sets}.

\header{Theorem 3}
$\C_{24}$ is 12-dimensional.
\begin{proof}
Let $\dim(\C_{24}) = k$ and take an element $v_{-2, 0, 2, 3}$ of $\C_{24}$ (where $v_S$ denotes the vector with 1s in the places of elements of $N_S$ and 0s elsewhere).
Then
\begin{align*}
  v_{-2} + v_0 + v_2 + v_3 = v_{-2, 0, 2, 3} = 011111001001010000000000
\end{align*}
This representation shows the first coordinate as $\infty$ and then each following coordinate for $0, 1, \dots, 22$ respectively.

This is a $\C$-set with first nonzero entry in the 0th spot.
Shifting each of the digits forward $i$ places for each of $i \leq 10$ gives ten new $\C$-sets, each with least digit $i$.
They are are linearl independent and have a $0$ in the $\infty$th coordinate.
Thus $k \geq 12$.

$\C_{24}$ is generated by a $k$-element set, $S$, of vectors of the form $v_i$.
Check that $v_\Omega = v_N = 0$.
However, if $v_N = 0$, each of the sums $v_{N_i}$ must also be 0 since the $N_i$ are permutations of the coordinates of the summands of $v_N$. So for each $v_i \in S \sse \C_{24}$, we have
$$ v_{N_i} = \sum_{j \in N_i} v_j = 0 $$
The set of all $k$ of thesee linear relations is linearly independent, so $k \leq 24 - k$, and therefor $k \leq 12$.

Altogethr, $k \leq 12 \land k \geq 12 \implies k = 12$.
\end{proof}

Since $\C_{24}$ is a 12-dimensional vector space over $\F_{2}$, it has $2^{12} = 4096$ elements.
Conway showed that these elements have the weight distribution
$$ 0^1 8^{759} 12^{2576} 16^{759} 24^1. $$
This amounts to the fact that there is one element with no 1-coordinates, 759 with exacty eight 1-coordinates, and so on.

It turns out that vectors with eight 1-coordinates generate all of $\C_{24}$.
These vectors are known interhangably as octads, and collectively as $\C(8)$.
$\C(8)$ is a Steiner system $S(5,8,24)$, meaning that given the location of five 1-coordinates of an octad, the other three 1s are uniquely determined.
This fact makes the notation of a sextet noteworthy.
A \textbf{sextet} is a partition of $\Omega$ into six 4-element subsets, the union of any two of which is an octad.
Since $\C(8)$ is a Steiner system, the choice of one 4-element subset of $\Omega$ (called a \textbf{tetrad}) uniquely determines the entire partition.
This will be useful for constructing the automorphism group of the Leech lattice.

The \textbf{Mathieu group} $M_{24}$ is defined to be the automorphism group of the Golay code.

\end{document}

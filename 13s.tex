\documentclass{article}

\usepackage[paper=a4paper, verbose, centering, margin=1.3in]{geometry}
\usepackage[utf8]{inputenc}
\usepackage{amsmath}
\usepackage{amssymb}
\usepackage{amsthm}
\usepackage{centernot}
\usepackage{graphicx}
\usepackage{amsthm}
\usepackage{mathtools}

\renewcommand{\tt}[1]{\text{ #1 }}

\newcommand{\Z}{\mathbb{Z}}
\newcommand{\R}{\mathbb{R}}
\newcommand{\C}{\mathcal{C}}
\newcommand{\F}{\mathbb{F}}

\renewcommand{\i}{^{-1}}
\newcommand{\sse}{\subseteq}
\renewcommand{\ss}{\subset}
\newcommand{\degrees}{^\circ}
\newcommand{\ass}[1]{\left( #1 \right)}
\newcommand{\set}[1]{\left\{ #1 \right\}}

\newcommand{\Leech}{\Lambda_{24}}
\newcommand{\Golay}{\mathcal{C}_{24}}
\newcommand{\PL}{\textit{PL}}

\newcommand{\header}[1]{\noindent\textsc{#1.} }

\begin{document}

\begin{center}
  {\huge The Problem of the Thirteen Spheres;\\A Proof for Undergraduates}
  \\[1em]
  {\large --- Notes ---}
\end{center}

\vspace{2em}

\header{Kissing Problem}
How many non-overlapping spheres can simultaneously touch a sphere of the same size in $\R^3$?

\header{Twelve Spheres}
A configuration where at each vertex of a regular icosohedron enscribing the central sphere is a sphere, all the spheres touch the central sphere.

\header{Thirteen Spheres}
Are thirteen spheres possible?

Unfortunately, this is a geometry paper and doesn't really have anything to do with Number Theory.
So, I have to find something else.


\end{document}
